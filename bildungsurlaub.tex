\documentclass{scrartcl}

\usepackage[ngerman]{babel}
%\usepackage[T1]{fontenc}
\usepackage[EU1]{fontenc} % http://tex.stackexchange.com/a/28454/14470
%\usepackage{fontspec}  % "With XeLaTeX, use the fontspec package instead of fontenc" - http://tex.stackexchange.com/questions/28450/xetex-%C3%9F-compiles-as-ss-german-umlauts-work

%\usepackage{amssymb} % for \square
%\usepackage{bbding}
\usepackage{wasysym}     % \Box and \CheckedBox
\usepackage{eurosym}
\usepackage{tabu}
\usepackage{xspace}

\usepackage{hyperref}

\newcommand{\thetitle}{Antrag
%\title{Bar}

auf Anerkennung
einer Weiterbildungsveranstaltung
zur Bildungsfreistellung bzw. zum Bildungsurlaub}

\newcommand{\raisedrule}[2][0em]{\leaders\hbox{\rule[#1]{1pt}{#2}}\hfill}

\begin{document}
%\maketitle
\begin{Large}
\hspace{.25\textwidth}\begin{minipage}{.75\textwidth}
\begin{center}
\thetitle
\end{center}
\end{minipage}
\end{Large}
\vspace{5ex}

\section{Veranstalter}
\begin{tabu} to 0.85\textwidth { X X }
Name: &                              
            LinuxTag e.V.
            \\
Anschrift mit PLZ: &
            Berlin
            \\
Telefon und Fax (mit Vorwahl): &     
            030-
            \\
E-Mail:                      &       
            info@linuxtag.de
            \\
Ansprechperson:              &       
            Tobias Mueller,
            tobi@guadec.org,
            +491633928710
            \\
Planen und führen \textbf{Sie} die Bildungsveranstaltung durch? & 
    $\CheckedBox$ ja \vspace{5em} $\Box$ nein\\
\end{tabu}

\section{Bezeichnung der Veranstaltung (ggf. bitte mit erläuterndem Untertitel)}

GUADEC (GNOME Users and Developers European Conference)

      GNOME ist eine wichtige Komponente in freien Betriebssystemen (GNU/Linux),
      wird als Freie und Open Source Software in der "Offentlichkeit von
      Freiwilligen für die Allgemeinheit entwickelt.
      Sowohl die Entwickler als auch interessierte Benutzer treffen sich seit
      über 15 Jahren jährlich auf dieser Konferenz, die jedes Jahr in einem
      anderen Land stattfindet\footnote{vgl. \url{https://wiki.gnome.org/GUADEC/}}.
      Es geht sowohl um technische Aspekte der Software als auch um die
      politische Dimension Freier Software und deren Einfluss auf unsere
      Gesellschaft.

\section{Veranstaltungstermin}

von
\hspace{5em}
%
2016-08-11
%
\hspace{5em}
bis
\hspace{5em}
%
2016-08-17\xspace
.

Die Veranstaltung soll wiederholt stattfinden:

$\Box$
ja

$\CheckedBox$
nein

Die Konferenz wird auch n"achstes Jahr stattfinden,
aber wir beantragen lediglich die Anerkennung f"ur dieses Jahr.

\section{Veranstaltungsort}
\begin{tabu} to  0.95\textwidth { X X[2] }
Anschrift mit PLZ:               &
    Karlsruhe Insititute of Technology~(KIT), Kaiserstraße 12, 76131 Karlsruhe
    \\
Telefon und Fax (mit Vorwahl):   &
    +491633928710\\
\end{tabu}



\section{Erst- oder Wiederholungsantrag}
Dies ist ein

\begin{enumerate}
\item[a)] $\Box$  Erstantrag

\item[b)] $\Box$  Wiederholungsantrag

          \begin{tabu} to \textwidth {X X}
          Aktenzeichen:      &       \\
          Es liegen wesentliche Änderungen gegen"uber dem Erstantrag vor,
            und zwar:   &   $\Box$ keine \\
            
          \end{tabu}

\item[c)]

    Die geplante Veranstaltung ist bereits anerkannt in folgenden 
    Ländern (bitte Bescheide beifügen):

    keine

\end{enumerate}


\section{Programm}
Das vollständige Programm%
\footnote{Diese Unterlagen/Angaben sind grundsätzlich beizufügen (Ausnahme Wiederholungsanträge)}
ist beigefügt. Aus ihm ist insbesondere ersichtlich
\begin{itemize}
\item das Lernziel der Veranstaltung
\item die Themen und Inhalte der einzelnen Unterrichtseinheiten
\item die detaillierte zeitliche und didaktisch-methodische Arbeitsplanung
\end{itemize}

\subsection{Welchem Bereich der Weiterbildung ordnen Sie die Veranstaltung zu?}

politische- und berufliche Bildung

%~\raisedrule{1pt}

%~\raisedrule{1pt}

\subsection{%
Die Veranstaltung ist keinem geschlossenen bereits vorab bekannten 
Personenkreis vorbehalten}

$\CheckedBox$ ja
\hspace{10em}
$\Box$ nein (bitte begründen)



\subsection{%
Angaben über die fachliche und pädagogische Qualifikation der Kursleitung und des
Lehrpersonals}


Sie finden anbei das Programm von letztem Jahr.
Dieses Jahr wird dem letzten Programm ähnlich sein, nur mit aktualisierten

\section{Teilnehmende}
\begin{tabu} to .95\textwidth {X X}
Vorgesehene Zahl der Teilnehmenden:     &  300  \\
Teilnahmeentgelt je Person:             &  \EUR{0} (Student) bis \EUR{150} (Professional), je nach Ticket
\end{tabu}

\section{Ist die Veranstaltung allgemein zugänglich?}
$\CheckedBox$ ja
\hspace{10em}
$\Box$ nein, Zielgruppen: \raisedrule{1pt}

\section{Wie wird die Veranstaltung öffentlich bekannt gemacht? (Bitte Unterlagen beifügen)}
Im Web auf \url{http://www.guadec.org} und \url{http://www.gnome.org}.

\section{Sind Sie mit der Veröffentlichung der Veranstaltung nach der Anerkennung durch
die Behörde einverstanden?}
$\CheckedBox $ja
\hspace{10em}
$\CheckedBox$ nein

\iffalse
\section{Geb"uhren}
Gilt für Länder mit Gebührenerhebung (z. Z. Hamburg und Sachsen-Anhalt):
Die Gebühr in Höhe von \EUR{0} wurde überwiesen an
\fi

\section{Versicherung}

Die vorstehenden Angaben sind richtig und vollständig. Nach 
Antragstellung eintretende Veränderungen werden unverzüglich mitgeteilt. 
Die Bildungsveranstaltung dient weder unmittelbar der Durchsetzung 
politischer Ziele noch ausschließlich betrieblichen oder dienstlichen 
Zwecken. Die Ziele des Veranstalters und der Inhalt der 
Bildungsveranstaltung stehen mit der freiheitlich demokratischen 
Grundordnung i. S. des Grundgesetzes für die Bundesrepublik Deutschland 
in Einklang.

Die Leitung der Veranstaltung ist einer Person unterstellt, die den 
Teilnehmenden namentlich bekannt ist.

\vspace{10ex}

~\raisedrule{1pt}

Ort, Datum, rechtsverbindliche Unterschrift

\end{document}

